% silica presentation
% Jacob Peck
% Quest 2012
% 20120319

\documentclass[12pt]{beamer}
\usetheme{Warsaw}
\usepackage{bera}
%\usepackage[all]{xy}

\definecolor{ttcolor}{RGB}{0,102,153}

% redefinition of \texttt
\let\oldtexttt\texttt
\renewcommand{\texttt}[2][ttcolor]{\textcolor{#1}{\ttfamily #2}}% \texttt[<color>]{<stuff>}



\begin{document}

\title{silica - a musical modeling language}
\author{Jacob M. Peck}
\date{April 18, 2012\\SUNY Oswego\\Quest 2012}
\maketitle

\section{Inspiration}
\subsection{Professor Craig Graci's Clay/MxM}
\begin{frame}{Professor Craig Graci's Clay/MxM}
  \begin{itemize}
    \item A software package used to develop musical intuitions
    \item A programming language to model music
    \item Used in Professor Graci's Cog 316/CSC 490 - Cognitive Musicology
  \end{itemize}
\end{frame}

\subsection{What is silica?}
\begin{frame}{What is silica?}
  \begin{itemize}
    \item ``Clay++''
    \item A superset of Clay, adding some new features
    \begin{itemize}
      \item User-defined scales and modes
      \item Functions
      \item Scripting
      \item Open source
    \end{itemize}
  \end{itemize}
\end{frame}

\subsection{What is siren?}
\begin{frame}{What is siren?}
  \begin{itemize}
    \item \textbf{SI}lica \textbf{R}endering \textbf{EN}gine
    \item A sonic and graphical renderer for silica
  \end{itemize}
\end{frame}

\section{A Guided Tour}
\subsection{The Note}
\begin{frame}{The Note}
  \begin{itemize}
    \item A global object, operated upon by primitives
    \item Consists of various parameters:
    \begin{itemize}
      \item scale degree
      \item octave (register, location)
      \item scale stack
      \item duration
      \item volume
      \item tempo
      \item instrument
    \end{itemize}
  \end{itemize}
\end{frame}

\subsection{Primitives}
\begin{frame}{Primitives}
  \begin{itemize}
    \item Act upon the note directly, manipulating one or more parameters
    \item A sampling:
    \begin{description}
      \item[play] Plays the note
      \item[rest] Rests the note
      \item[rp and lp] Raise the pitch and lower the pitch
      \item[x2, x3, x5, x7] E\textbf{x}pand the duration by a factor
      \item[s2, s3, s5, s7] \textbf{S}hrink the duration by a factor
      \item[inctempo] Increase the tempo by 10 BPM
      \item[decvol] Decrease the volume by 100 (MIDI value)
      \item[agogo] Change the instrument to agogo
      \item[d-minor] Change the scale to Dm.
    \end{description}
  \end{itemize}
\end{frame}

\subsection{The rest of the good stuff}
\begin{frame}{The rest of the good stuff}
  \begin{description}
    \item[Meta Commands] Interact with silica at a level higher than the note.  \\Ex. \textit{\texttt{-state}} or \textit{\texttt{-exit}}
    \item[Macros] Simple rewrite rules.  \\Ex. \textit{\texttt{asc3 >> 3play+rp 3lp}}
    \item[Commands] Macros with embedded structure.
    \item[Functions] Macros with replacement rules.  \\Ex. \textit{\texttt{hat(x) := x rp x lp x}}
    \item[Transforms] Manipulations on a Macro / Function / Command.
  \end{description}
\end{frame}

\section{End}
\subsection{Demo time?}
\begin{frame}{Demo time?}
  \begin{itemize}
    \item I'm sure you're probably confused at this point.
    \item One second while I set things up...
  \end{itemize}
\end{frame}

\subsection{Any questions?}
\begin{frame}{Any questions?}
  \begin{itemize}
    \item Feel free to ask.
  \end{itemize}
\end{frame}

\subsection{Where to get it}
\begin{frame}{Where to get it}
  \url{http://silica.suspended-chord.info/} \\
  and\\
  \url{https://github.com/gatesphere/silica}
\end{frame}

\end{document}
